\documentclass{article}

%   included packages
\usepackage[a4paper,margin = 14mm]{geometry} %0.5 inch margins a4 paper
\usepackage{amsmath}
\usepackage{newpxtext,newpxmath}
\usepackage{microtype}
\usepackage{titling}

%   version history (should be used in conjunction with git)
\usepackage{vhistory}

%   Allows me to use hyperlinks for table of contents etc.
\usepackage{hyperref}
\hypersetup{
    linktoc=all,
    colorlinks,
    citecolor = black,
    filecolor = black,
    linkcolor = black,
    urlcolor = blue
}

%   Setting up the code example
% \usepackage{listings}
\usepackage{xcolor}


% \definecolor{dkgreen}{rgb}{0,0.6,0}
% \definecolor{gray}{rgb}{0.5,0.5,0.5}
% \definecolor{mauve}{rgb}{0.58,0,0.82}

% \lstset{frame=tb,
%   language=C++,
%   aboveskip=3mm,
%   belowskip=3mm,
%   showstringspaces=false,
%   columns=flexible,
%   basicstyle={\small\ttfamily},
%   numbers=none,
%   numberstyle=\tiny\color{gray},
%   keywordstyle=\color{blue},
%   commentstyle=\color{dkgreen},
%   stringstyle=\color{mauve},
%   breaklines=true,
%   breakatwhitespace=true,
%   tabsize=3
% }

%   formatting for document

%   Push everything to the left cause I think it looks cleaner
\preauthor{\begin{flushleft}}
\postauthor{\end{flushleft}}
\predate{\begin{flushleft}}
\postdate{\end{flushleft}}    
\setcounter{secnumdepth}{0}

%   Document Title & author
\title{\Huge Multi-Programming Analogy}
\author{By: Anthony Sasso}

%   updates date for each compile, that way this + versioning tells reader which they are reading?
\date{\today}

%no indents for paragraphs (change depending on doc type)
\setlength{\parindent}{0pt}

% custom commands / formatting shortcuts
%   combine bold & italic
\newcommand{\textbfit}[1]{\textbf{\textit{#1}}} 

%   footnote with label, colour, and inputted text... then sets back to default
\newcommand{\clfootnote}[3]{\color{#2}{\footnote{\label{#1}{ \color{#2}{#3}}}}\color{defaultcolor}} 

%   set to defualt colour (black)
\AtBeginDocument{\colorlet{defaultcolor}{.}}

%   writing goes here
\begin{document}
    
    \maketitle

    \begin{versionhistory}
        %   Note this is in YYYY/MM/DD cause it is closer to GMT and makes more sense to me, probably don't need to prefice since last options are >12 so implied middle are months?

        \vhEntry{1.5}{2022/03/21}{Anthony Sasso}{converted to .tex from .rtf, changed formatting but no core text altered yet}
        \vhEntry{1}{2021/02/25}{Anthony Sasso}{original rtf this is from lists last modification at this time (not technically possible but didn't use version history originally\dots)}
        \vhEntry{0.5}{2021/09/30}{Anthony Sasso}{created}
    \end{versionhistory}

    \tableofcontents

    \newpage
    \section{Introduction / Concept}
    
    Just an analogy I had said in class for understanding multiprogramming terms that I thought I should write down for later\dots

    \bigskip
    Essentially ``One way to understand the differences between processes, threads, semaphore, and mutex is media like Azure / GitHub, Google Docs, email, and a punch out card.''.

    \section{Multiprocessing}
    
    Multiprocessing operates fundamentally separately with the fork() command creating a segmented process with identical data (which can then merge / push using the pipe() command). This is relatable to how Azure and GitHub creates a clone / local copy that can be manipulated until it is merged back, overwriting the data in the parent or main directory.

    \section{Multi-Threading}

    Threads somewhat similarly are shared data with multiple ``actors'' manipulating and altering the information. This can be seen analogously in `Google Docs' with multiple people all writing to the same document, with each of them being unique, separately operating functions with a shared resource pool between them.

    \section{Semaphores}

    A semaphore differently can only send a ``flag'' to other linked processes potentially updating, altering, or alerting them. Think of this as an employee waiting for their boss to give them the go-ahead email, not able to continue until this is reached; or that same employee receiving notification they must stop production of a new product immediately.

    \section{Mutexes}
    
    Mutexes are the ``hard-coded'' alternative of semaphore. Think of it like a punch out card where the halted program cannot and will not continue until another program (the one who initiated the mutex) “punches out” and lets it continue.

\end{document}
